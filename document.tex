\documentclass[aspectratio=169]{beamer}
\usepackage[utf8]{inputenc}
\usepackage{multicol}

\title{Towards Pulverised Architectures for Collective Adaptive Systems through Multi-tier Programming}

\author[G.Aguzzi]{
  \textbf{Gianluca Aguzzi}\inst{1} \and
  Roberto Casadei\inst{1} \and
  Danilo Pianini\inst{1} \\
  Guido Salvaneschi\inst{2} \and
  Mirko Viroli\inst{1}
}
\institute{
  \inst{1}
  \texttt{Alma Mater Studiorum} -- Università di Bologna, Cesena, Italy \\
  \inst{2}
  University of St.Gallen: St.Gallen, Switzerland
}

\usetheme{material}

\useLightTheme
\usePrimaryTeal
\useAccentIndigo

\begin{document}

\begin{frameImg}{img/background.jpg}
\titlepage
\end{frameImg}

%\begin{frame}{Table of contents}
%\begin{card}
%\tableofcontents
%\end{card}
%\end{frame}

\section{Overview}
\begin{frameImg}{img/background.jpg}
\begin{card}[Collective Adaptive Systems]
\textit{a definition}
\end{card}
\end{frameImg}

\begin{frame}{Design problems (partial)}
\begin{card}
\begin{itemize}
  \item<1-| alert@1> A
  \item<2-| alert@2> B
  \item<3-| alert@3> C
\end{itemize}
\end{card}
\end{frame}

\section{Proposed approach}
\begin{frame}{Proposed approach}
\begin{multicols}{3}
[
  \begin{cardTiny}
  Todos
  \end{cardTiny}
]

\centering
\cardImg{example-image-a}{0.3\textwidth} 
\pause
\cardImg{example-image-b}{0.1\textwidth}
\cardImg{example-image-b}{0.3\textwidth}
\end{multicols}
\end{frame}

\section{Aggregate Computing}
\begin{frame}{Aggregate Computing}
\begin{card}
{\color{accent} \textit{Short definition} }
\end{card}
\centering
\cardImg{example-image-b}{0.4\textwidth}
\end{frame}

\section{Pulverisation}
\begin{frame}{Pulverisation}
\begin{multicols}{2}
[
  \begin{card}
  {\color{accent} \textit{Short definition} }
  \end{card}
]
\centering
\cardImg{example-image-a}{0.3\textwidth} 
\pause
\cardImg{example-image-b}{0.3\textwidth}
\end{multicols}
\end{frame}
\section{Multi-tier programming}

\begin{frame}{Multi-tier programming}

\end{frame}

\section{Framework choosen}
\begin{frame}{Frameworks}
\begin{card}[ScaFi]
\end{card}
\centering
\begin{card}[ScalaLoci]

\end{card}

\end{frame}

\end{document}
