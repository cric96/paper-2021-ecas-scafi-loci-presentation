\documentclass[aspectratio=169]{beamer}
\usepackage[utf8]{inputenc}
\usepackage{multicol}
\usepackage{tikz}
\usepackage[backend=bibtex,sorting=none,style=numeric,doi=true]{biblatex} 
\usepackage{adjustbox}

\title{Towards Pulverised Architectures for Collective Adaptive Systems through Multi-tier Programming}

\author[G.Aguzzi]{
  \textbf{Gianluca Aguzzi}\inst{1} \and
  Roberto Casadei\inst{1} \and
  Danilo Pianini\inst{1} \\
  Guido Salvaneschi\inst{2} \and
  Mirko Viroli\inst{1}
}

\institute{
  \inst{1}
  \texttt{Alma Mater Studiorum} -- Università di Bologna, Cesena, Italy \\
  \inst{2}
  University of St.Gallen: St.Gallen, Switzerland
}


\usetheme{material}
\useLightTheme
\usePrimaryTeal
\useAccentIndigo

%% bib
\bibliography{biblio.bib}

\begin{document}

\begin{frameImg}{img/background.jpg}
  \titlepage
\end{frameImg}

%\begin{frame}{Table of contents}
%\begin{card}
%\tableofcontents
%\end{card}
%\end{frame}

\section{Overview}
  \begin{frameImg}{img/background.jpg}
  \begin{card}[Collective Adaptive Systems]
    {
      \color{accent} Refer to a form of complex systems where 
      a \textit{large number} of \textit{heterogeneous} entities interact without specific \textit{external} or \textit{internal} 
      central control, adapt their behaviour to environmental 
      settings in pursuit of an \textit{individual} or \textit{collective} goal.}\footnote{\url{http://unige.ch/cui/cas/}
    } 
  \end{card}
\end{frameImg}

\begin{frame}{Challenges}
%\only<1-4>{ \begin{backgroundblock} \includegraphics[width=\paperwidth]{img/network.jpg} \end{backgroundblock} }
%\only<5-6>{ \begin{backgroundblock} \includegraphics[width=\paperwidth]{example-image-a} \end{backgroundblock} }
  \only<1-4>{ \begin{backgroundblock} \includegraphics[width=\paperwidth]{img/complex-network.png} \end{backgroundblock} }
  \only<5-6>{ \begin{backgroundblock} \includegraphics[height=\paperheight]{img/swarms.png} \end{backgroundblock} }

  \begin{card}
  { 
    \setbeamercovered{transparent=10}
    \begin{itemize}
      \item<1-| alert@1> Complex and layered networks
      \begin{itemize}
        \item <2-| alert@2>Large scale
        \item <3-| alert@3>Heterogenous
        \item <4-| alert@4>Dynamic
      \end{itemize}
      \item<5-| alert@5> Distributed control
      \item<6-| alert@6> Environmental changes
    \end{itemize}
  }

  \end{card}
\end{frame}

\section{Proposed approach}
\begin{frame}{Proposed approach}

\begin{cardTiny}
{ 
  \setbeamercovered{transparent=10}
  \begin{itemize}
    \item<1-> Describe the collective behaviour with {\color{accent} \textit{Aggregate Computing}~\cite{DBLP:journals/jlap/ViroliBDACP19}}
    \item<2-> Define your concrete deployment with {\color{accent} \textit{Multi-tier programming}~\cite{DBLP:journals/csur/WeisenburgerWS20}}
    \item<3-> Inject the {\color{accent} \textit{same}} collective behaviour in different deployments thanks to {\color{accent} \textit{Pulverisation}~\cite{DBLP:journals/fi/CasadeiPPVW20}}
  \end{itemize}
}
\end{cardTiny}
\begin{columns}[onlytextwidth, t]
  \begin{column}{0.27\textwidth}
    \begin{adjustbox}{max width=\textwidth, valign=m, margin=0}
      \cardImg{img/aggregate-computing-file}{\textwidth}
    \end{adjustbox}
  \end{column}
  \begin{column}{0.32\textwidth}
    \begin{adjustbox}{max width=\textwidth, valign=m}
      \presentationGraphics{img/scalaloci}{1}{1}
    \end{adjustbox}
  \end{column}
  \begin{column}{0.25\textwidth}
    \begin{adjustbox}{max width=\textwidth, valign=m}
      \presentationGraphics{img/finalidea}{1-2}{1}
    \end{adjustbox}
  \end{column}
\end{columns}
%\presentationGraphics{img/aggregate-computing-file}{1-2}{0.2}
\end{frame}

\section{Aggregate Computing}
\begin{frame}{Aggregate Computing~\cite{DBLP:journals/jlap/ViroliBDACP19}}
  \begin{cardTiny}
    {
      \color{accent} A top-down paradigm by wich programmers describe 
      a collective behaviour in a declarative and functional way,
      manipulating a distributed data structure called \textit{Computational field} 
    }
  \end{cardTiny}
  \centering
  \cardImg{img/aggregate-computing.png}{0.6\textwidth}
\end{frame}

\section{Pulverisation}
\begin{frame}{Pulverisation~\cite{DBLP:journals/fi/CasadeiPPVW20}}

{
  \setbeamercovered{transparent=10}  
    \begin{cardTiny}
      {
        \color{accent} An approach proposed for Aggregate Computing 
        to neatly separate behavioural and deployment concerns. 
      }
    \end{cardTiny}
    \begin{columns}
      \begin{column}{0.45\textwidth}
        \presentationGraphics{img/logical-system.jpg}{1}{1}
      \end{column}
      \begin{column}{0.38\textwidth}
        \presentationGraphics{img/deployments}{1-2}{1}
      \end{column}
    \end{columns}
    \only<3>{} 
}
\end{frame}

\section{Multi-tier programming}

\begin{frame}{Multi-tier programming~\cite{DBLP:journals/csur/WeisenburgerWS20}}
  \begin{cardTiny}
  {
    \color{accent} A programming apporach by wich distributed architectures 
    are defined in a single compilation unit with a single language. 
  }
  \end{cardTiny}
  \centering
  \cardImg{img/multitier}{0.4\textwidth}
\end{frame}

\section{Framework choosen}
\begin{frame}{Frameworks}%

{
  \setbeamercovered{transparent=10}
  \begin{card}[ScaFi~\cite{DBLP:conf/isola/CasadeiVAD20} (Scala Field)]
    A tool-chain composed by a Domain Specific Language that supports field-calculus.
  \end{card}
  \pause
  \begin{card}[ScalaLoci~\cite{Weisenburger.2018}]
    A langauge extension to enable compile-time multi-tier programming.
  \end{card}
}
\end{frame}

\section{Puliverised architectures}
\begin{frame}{Puliverised architectures: \textbf{logical components}}

\end{frame}
\begin{frame}{Puliverised architectures: \textbf{concrete deployments}}
\end{frame}

\section{Benefits}
\begin{frame}{Benefits}
\end{frame}


\begin{frame}[allowframebreaks]
\frametitle{References}
%\bibliographystyle{abbrv}
%\bibliography{biblio} 
\printbibliography
\end{frame}

\end{document}
